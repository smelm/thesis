\section*{Zusammenfassung}

Reguläre Ausdrücke sind ein mächtiges und essentielles Werkzeug in der modernen Softwareentwicklung aber auch Quelle von Programmierfehlern und Wartbarkeitsproblemen. Ihre schlechte Lesbarkeit und unintuitive Semantik sind selbst für erfahrene Nutzer eine Belastung und eine unüberwindbare Barriere für Beginner. Es ist nicht ungewöhnlich, dass Entwickler Regex aus anderen Quellen kopieren, was aufgrund der Inkompatibilität zwischen Programiersprachen problematisch ist. Diese Abschlussarbeit untersucht die Probleme von regulären Ausdrücken und wie diese in der Praxis verwendet werden. Auf diesen Erkenntnissen basierend, wird die Domänen spezifische Sprache (DSL) RexRegex entworfen, die eine interne und eine externe DSL vereint, um die Lücke zwischen Programmieren und Nichtprogrammieren zu schließen. Um die Korrektheit des generierten Regex über Kompilationsziele hinweg sicherzustellen, wird das Testframework UTGAST (Unit Test Generating Abstract Syntax Tree) vorgestellt. Außerdem wird eine Mechanismus präsentiert, der es mithilfe eines genetischen Algorithmus erlaubt, DSL Code von Eingabebeispielen zu generieren.
