\chapter{Evaluation}

This work presents the domain specific language RexRegex and the \textsc{utgast} framework in an attempt to address the issues of regular expressions. This chapter will evaluate whether it succeeded.

\section{Drawbacks}

The popularity of regular expressions partially stems from the fact that they offer a lot of expressive power in a compact form. RexRegex compiles to regex which makes it the upper bound of concepts that can be expressed in the language. Since there are regex constructs that were excluded from the language, it is out-of-the-box strictly less powerful then regex. This is mitigated by its macro and extension system which allows access to the more advanced regex features.

The exclusion of regex features also causes difficulty when interfacing with other tools. Features that are not supported can not be expressed in the RexRegex AST. This leads to incompatibilities with the RegexGenerator \cite{bartoli2016inference} when implementing the inference from input examples (see section \ref{sec:inferenceOfRexRegex}). These challenges are to be expected when interfacing with any type of library or when attempting to directly translate regex into the DSL.

The language was intentionally designed to be verbose, which is supposed to maximize readability. It is likely that a portion of users -- especially developers -- perceive it as too verbose. While the intent behind this may be appreciated, it may still feel like typing a lot while achieving little. In this regard, RexRegex is in a similar position as Java, which uses a more verbose syntax for more readability compared to the C language and has been widely criticized for it. The verbosity of the language also excludes it from some use cases of regex like searching in editors and search engines where the compact nature of them shines the most.

While the language is an indisputable improvement in readability and maintainability, the regex it generates are harder to read than human-written ones. It is the case for most compilers. This has the consequence that mistakes in the compiler and misuse of language features are a lot harder to debug.

In his lecture \enquote{On Successful Language Design} \cite{OnSuccessfulLanuageDesignKernighan} Brian Kernighan notes that the two key factors for the success of a language are a weak competition and good luck. As laid out in section \ref{sec:relatedWork} there are already DSLs that address regular expressions, some even backed by major companies like Microsoft or Apple. While none solve the portability issues of regex -- and by extensions the regex DSLs -- it would be hard to establish a new DSL as the standard.

\section{Benefits}
 
RexRegex code is very readable. Its keywords and syntax algin with the English language and allow a very intuitive user experience. The exclusion of procedural concepts and high-level abstractions allow for a very declarative style. These factors make RexRegex an ideal tool for non-programmers who would appreciate its beginner-friendly syntax and self-explanatory nature. It is reasonable to expect that even without prior programming knowledge someone would be able to use RexRegex after a short learning period. Whenever where this is not the case, the inference mechanism provides a starting-of-point for the user.

The internal DSL allows a seamless integration into the host language which appeals to developers. It allows them to leverage IDE features for easy access to type information and documentation, while being more maintainable than regex. It strikes a good balance between readability and ease of use.

The DSL can easily be expanded to compile to additional regex dialects. The test suite can be applied to the new compilation target with little overhead and guarantees that the DSL is consistent across languages. This represents a large step towards solving the portability problem and creating a \enquote{universal translator} for regex \cite{RegexNotLinguaFranca}.

The macro and extension system allow the language to grow in a maintainable fashion. It provides a way for the user base to expand the language in a scaleable way. The integration with the \textsc{utgast} framework makes it easy for contributors to ensure that the additions are well-tested and that every new extension is compatible with all compilation targets. Extensions allow the user to mold the DSL to their specific domain and create abstractions for a more declarative style and readability. There is a foreseeable future where matching an email would only be an import statement away, while the user can rest assured that the pattern was verified for the programming language they use. This scenario is a vast improvement over the current state of affairs where regular expressions are copied regularly from online sources with a significant likelihood that they may function differently than intended \cite{RegexNotLinguaFranca}.

While the success of a language is multi-faceted, it is justified to say that RexRegex fulfills the necessary requirements: it \enquote{solve[s] a real problem in a clearly better way} \cite{OnSuccessfulLanuageDesignKernighan}.

\section{Further Work}

There is still room for improvement in RexRegex. Extending the standard library is one of the priorities. Another one is to improve the error reporting during parsing to make it more user-friendly. It would also be beneficial to implement a linting mechanism that detects valid DSL that compiles to problematic patterns. This can include ranges like \code{"A" to "z"} that are most likely wrong or even super-linear patterns.

Because of the inference (see section \ref{sec:inferenceOfRexRegex}) RexRegex does not only parse DSL strings into ASTs but also pretty-prints ASTs into DSL strings. This represents a lot of duplicated logic and it may benefit the long-term maintainability to use some sort of bilateral grammar. While there is research on this topic \cite{BidirectionalGrammarsForMachineCode}\cite{GuidedGrammarConvergence}, there are still a lot of open questions in this regard.

The \utgast{} framework offers huge potential. In the context of RexRegex it could be expanded to not just generate sample strings but also the groups that are expected to match. Another possible improvement would be to allow the framework to generate samples for substring matches, which would be particularly useful when lazy quantifiers were included in the language.

There are also possible applications for \utgast{} that lie outside of the scope of RexRegex. The principle could be applied to any DSL that 1. has an AST and 2. compiles to more than one target. One particular example would be DSLs for data processing that compile to different SQL dialects and/or frameworks like Pandas. The test generation would need to be extended to include setup and teardown logic and assert table content instead of string matches but it is more than feasible. 

There is still a lot of work to be done but the RexRegex language and the \utgast{} framework offer the potential to solve a lot of the issues of regular expressions.
