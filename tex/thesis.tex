% This template was initially provided by Dulip Withanage.
% Modifications 
% were made by Conny Junghans and Jannik Strötgen
% and Sascha Hunold

\documentclass[
     12pt,                    % font size
     a4paper,             % paper format
     BCOR10mm,     % binding correction
     DIV14,                 % stripe size for margin calculation
     listof=totoc,                    % table listing in toc
     bibliography=totoc,       % bibliography in toc
     index=totoc,              % index in toc
%     parskip            % paragraph skip instead of paragraph indent
     twoside,
     headsepline
     ]{scrreprt}

%%%%%%%%%%%%%%%%%%%%%%%%%%%%%%%%%%%%%%%%%%%%%%%%%%%%%%%%%%%%


% PACKAGES:

% Use German :
% make sure texlive-lang-german is installed (for TexLive)
\usepackage[ngerman]{babel}
% Input encoding
\usepackage[utf8]{inputenc}
% Font encoding
\usepackage[T1]{fontenc}
% Index-generation
\usepackage{makeidx}
% Einbinden von URLs:
\usepackage{url}
% Special \LaTex symbols (e.g. \BibTeX):
\usepackage{doc}
% Include Graphic-files:
\usepackage{graphicx}

% Fuer anderthalbzeiligen Textsatz
\usepackage{setspace}

% hyperrefs in the documents
\usepackage[bookmarks=true,colorlinks,pdfpagelabels,pdfstartview = FitH,bookmarksopen = true,bookmarksnumbered = true,linkcolor = black,plainpages = false,hypertexnames = false,citecolor = black,urlcolor=black]{hyperref} 


%%%%%%%%%%%%%%%%%%%%%%%%%%%%%%%%%%%%%%%%%%%%%%%%%%%%%%%%%%%%

% OTHER SETTINGS:

% Pagestyle:
\pagestyle{headings}

% Choose language
\newcommand{\setlang}[1]{\selectlanguage{#1}\nonfrenchspacing}


\begin{document}

% TITLE:
\pagenumbering{roman} 
\begin{titlepage}


\vspace*{1cm}
\begin{center}
\vspace*{3cm}
\textbf{ 
\Large Ruprecht-Karls-Universität Heidelberg\\
\smallskip
\Large Institut für Informatik\\
\smallskip
\Large Lehrstuhl für Parallele und Verteilte Systeme\\
\smallskip
}

\vspace{3cm}

\textbf{\large Bachelorarbeit}

\vspace{0.5\baselineskip}
{\huge
    A DSL for Regex
}
\end{center}

\vfill 

{\large
\begin{tabular}[l]{ll}
Name: & Samuel Melm\\
Matrikelnummer: & 4068750 \\
Betreuer: & Prof. Dr. Andrzejak\\
Datum der Abgabe: & dd.mm.yyyy
\end{tabular}
}

\end{titlepage}

\onehalfspacing

\thispagestyle{empty}

\vspace*{100pt}
\noindent
Ich versichere, dass ich diese Bachelor-Arbeit selbstständig verfasst und nur die angegebenen
Quellen und Hilfsmittel verwendet habe.

\vspace*{50pt}

\noindent
Heidelberg, dd.mm.yyyy
\cleardoublepage

\section*{Zusammenfassung}

Dies ist eine Zusammenfassung der Arbeit.

\section*{Abstract}

This is the abstract.

\cleardoublepage

\tableofcontents
\cleardoublepage
\pagenumbering{arabic} 

% List of figures (Abbildungsverzeichnis):
\listoffigures
% List of tables (Tabellenverzeichnis):
\listoftables
\cleardoublepage

%%%%%%%%%%%%%%%%%%%%%%%%%%%%%%%%%%%%%%%%%%%%%%%%%%%%%%%%%%%

\chapter{Einleitung}\label{intro}
This chapter contains an overview of the topic as well as the goals and contributions of your work.

\section{Motivation}

\section{Ziele der Arbeit}

\section{Aufbau der Arbeit}
Here you describe the structure of the thesis. For example:

In Kapitel~\ref{background} werden grundlegende Methoden für diese Arbeit vorgestellt.


\chapter{Grundlagen}\label{background}

\section{Scheduling}
Here you discuss some basics for your work and outline existing research in the area of your thesis by citing research papers like~\cite{BCFLMR_TPDS08} by Beaumont \textit{et al.} or~\cite{legrand_ccgrid03,TelBook}.

Außerdem sind in Tabelle~\ref{tab:foobar} Studenten aufgeführt, die besonders erwähnenswert gearbeitet haben.

\begin{table}[htb]
\centering
\captionabove{\label{tab:foobar}Ausgezeichnete Studenten}
\begin{tabular}{ll}
\hline
Name & Vorname \\
\hline
Müller  &  Horst \\
Meyer & Fred \\
\hline
\end{tabular}
\end{table}

Will man auf eine Abbildung verweisen, kann man das folgendermaßen tun, siehe Abbildung~\ref{fig:foobar}.
Man sollte aber nicht vergessen, diese Grafik zu beschreiben und ggf. auf die Ursprungsquelle hinzuweisen, wenn die Grafik nicht vom Autoren selbst erstellt wurde.

\begin{figure}[htb]
\centering
\includegraphics[width=.4\linewidth]{vert}
\caption[Kurzname der tollen Grafik]{\label{fig:foobar}Das ist eine tolle Grafik.}
\end{figure}

\clearpage
Hier sind noch mehr Grundlagen.

\bibliographystyle{plain}
\bibliography{references}

\end{document}
