
\chapter{Einleitung}\label{intro}
This chapter contains an overview of the topic as well as the goals and contributions of your work.

\section{Motivation}

\section{Ziele der Arbeit}

\section{Aufbau der Arbeit}
Here you describe the structure of the thesis. For example:

In Kapitel~\ref{background} werden grundlegende Methoden für diese Arbeit vorgestellt.


\chapter{Grundlagen}\label{background}

\section{Scheduling}
Here you discuss some basics for your work and outline existing research in the area of your thesis by citing research papers like~\cite{BCFLMR_TPDS08} by Beaumont \textit{et al.} or~\cite{legrand_ccgrid03,TelBook}.

Außerdem sind in Tabelle~\ref{tab:foobar} Studenten aufgeführt, die besonders erwähnenswert gearbeitet haben.

\begin{table}[htb]
\centering
\captionabove{\label{tab:foobar}Ausgezeichnete Studenten}
\begin{tabular}{ll}
\hline
Name & Vorname \\
\hline
Müller  &  Horst \\
Meyer & Fred \\
\hline
\end{tabular}
\end{table}

Will man auf eine Abbildung verweisen, kann man das folgendermaßen tun, siehe Abbildung~\ref{fig:foobar}.
Man sollte aber nicht vergessen, diese Grafik zu beschreiben und ggf. auf die Ursprungsquelle hinzuweisen, wenn die Grafik nicht vom Autoren selbst erstellt wurde.

\begin{figure}[htb]
\centering
\includegraphics[width=.4\linewidth]{vert}
\caption[Kurzname der tollen Grafik]{\label{fig:foobar}Das ist eine tolle Grafik.}
\end{figure}

\clearpage
Hier sind noch mehr Grundlagen.
