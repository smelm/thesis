\section*{Abstract}

Regular expressions are a powerful and essential tool in modern software development but also a source for errors and maintainability issues. Their poor readability and unintuitive semantics are a burden even to experienced users and present an insurmountable barrier for beginners. It is also common for developers to simply reuse regex from other sources, unware of the fact that their syntax and semantics differ from language to language. This work analyses the usage and issues of regular expressions in practice. Based on this analysis the domain specific language (DSL) RexRegex is developed which combines an internal and external DSL to bridge the gap between programmers and non-programmers. To validate the correctness regex output across compilation targets, the novel testing framework UTGAST (Unit Test Generating Abstract Syntax Tree) is introduced. It also presents a mechanism to infer DSL code from string examples based on a genetic algorithm.
