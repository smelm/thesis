\section*{Abstract}

Regular expressions are a powerful and essential tool in modern software development but also a source of errors and maintainability issues. Their poor readability and unintuitive semantics are a burden even to experienced users and a insurmountable barrier for beginners. It is common for developers to simply reuse regex from other sources which is problematic since they are not compatible between different programming languages. This work analyses the issues of regular expressions and how they are used in practice. Based on this analysis the domain specific language (DSL) RexRegex is developed which combines an internal and external DSL to bridge the gap between programmers and non-programmers. To ensure the correctness of the regex output across compilation targets the novel testing framework UTGAST (Unit Test Generating Abstract Syntax Tree) is introduced. It also presents a mechanism to infer DSL code from string examples based on a genetic algorithm.
