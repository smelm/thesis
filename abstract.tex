\section*{Abstract}

Regular expressions are a powerful and essential tool in modern software development but also a source for errors and maintainability issues. Their poor readability and unintuitive semantics are a burden even to experienced users and present an insurmountable barrier for beginners. It is also common for developers to simply reuse regex from other sources, unware of the fact that their syntax and semantics differ from language to language. This work analyses the usage and issues of regular expressions in practice. Based on this analysis the domain specific language (DSL) RexRegex is developed which combines an internal and external DSL to bridge the gap between programmers and non-programmers. To validate the correctness of the generated regex across compilation targets, the novel testing framework UTGAST (Unit Test Generating Abstract Syntax Tree) is introduced. It also presents a mechanism to infer DSL code from string examples based on a genetic algorithm.


\section*{Zusammenfassung}

Reguläre Ausdrücke sind ein mächtiges und essentielles Werkzeug in der modernen Softwareentwicklung aber auch Quelle von Programmierfehlern und Wartbarkeitsproblemen. Ihre schlechte Lesbarkeit und unintuitive Semantik sind selbst für erfahrene Nutzer eine Belastung und eine unüberwindbare Barriere für Beginner. Es ist nicht ungewöhnlich, dass Entwickler Regex aus anderen Quellen kopieren, was aufgrund der Inkompatibilität zwischen Programiersprachen problematisch ist. Diese Abschlussarbeit untersucht die Probleme von regulären Ausdrücken und wie diese in der Praxis verwendet werden. Auf diesen Erkenntnissen basierend, wird die domänenspezifische Sprache (DSL) RexRegex entworfen, die eine interne und externe DSL vereint, um die Lücke zwischen Programmieren und Nichtprogrammieren zu schließen. Um die Korrektheit des generierten Regex über Kompilationsziele hinweg sicherzustellen, wird das Testframework UTGAST (Unit Test Generating Abstract Syntax Tree) vorgestellt. Außerdem wird eine Mechanismus präsentiert, der es mithilfe eines genetischen Algorithmus erlaubt, DSL Code von Eingabebeispielen zu generieren.
